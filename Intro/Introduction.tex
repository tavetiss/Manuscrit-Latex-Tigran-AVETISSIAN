
% LTeX: language=fr
%%%%%%%%%%%%%%%%%%%%%%%%%%%%%%%%%%%%%%%%%%%%%%%%%%%%%%%%%%%%%%%%%%%%%%%%%%
%%%%%                        Intro Générale                         %%%%%%
%%%%%%%%%%%%%%%%%%%%%%%%%%%%%%%%%%%%%%%%%%%%%%%%%%%%%%%%%%%%%%%%%%%%%%%%%%
\phantomsection 
\addcontentsline{toc}{chapter}{Introduction générale}
\addtocontents{toc}{\protect\addvspace{10pt}}

\vspace*{-1cm}
\begin{flushright}
\section*{\fontsize{24pt}{24pt}\selectfont\textnormal{Introduction générale}}
\end{flushright}
\vspace{2cm}

\lhead[\fancyplain{}{Introduction générale}]
      {\fancyplain{}{}}
\chead[\fancyplain{}{}]
      {\fancyplain{}{}}
\rhead[\fancyplain{}{}]
      {\fancyplain{}{Introduction générale}}
\lfoot[\fancyplain{}{}]%
      {\fancyplain{}{}}
\cfoot[\fancyplain{}{\thepage}]
      {\fancyplain{}{\thepage}}
\rfoot[\fancyplain{}{}]%
     {\fancyplain{}{\scriptsize}}
     
     

%%%%%%%%%%%%%%%%%%%%%%%%%%%%%%%%%%%%%%%%%%%%%%%%%%%%%%%%%%%%%%%%%%%%%%%%%%
%%%%%                      Start part here                          %%%%%%
%%%%%%%%%%%%%%%%%%%%%%%%%%%%%%%%%%%%%%%%%%%%%%%%%%%%%%%%%%%%%%%%%%%%%%%%%%

\lettrine[lines=1]{L~}{}a science et la technologie ne cessent d'accélérer le développement et la miniaturisation des dispositifs nomades dans l'environnement du corps humain. Allant de la télémédecine aux transactions financières, en passant par les montres connectées et les appareils d'aide à l'audition. Ce contexte applicatif demande cependant une attention particulière à la bio-intégrabilité et à la miniaturisation des systèmes. En outre, ces-derniers nécessitent une alimentation électrique pour leur fonctionnement. Le développement croissant d'unités de stockage d'énergie tels que les batteries et les supercondensateurs apporte des solutions viables pour la plupart des applications.

L'éco-responsabilité et l'aspect financier à long terme motivent l'utilisation préférentielle des dispositifs rechargeables. La praticité et l'accès restreint aux sources électriques appellent alors au développement de systèmes de récupération d'énergie pour augmenter l'autonomie des appareils en visant, à terme, de les rendre entièrement autonomes.

Les travaux exposés dans cette thèse s'articulent plus particulièrement autour du besoin énergétique des appareils auditifs. Les fonctions primaires de ces dispositifs requièrent des puissances inférieures à 500µW pour les systèmes les plus optimisés. Le corps humain génère une quantité d'énergie non négligeable en comparaison. De nature chimiques, thermiques, ou bien mécaniques, ces sources d'énergie biologiques peuvent être exploités au travers de différentes technologies de transduction. Nous allons nous intéresser plus particulièrement à l'exploitation de l'énergie de déformation mécanique du conduit auditif. Ce phénomène découle des mouvements de la mâchoire. Le joint temporomandibulaire, liant la mâchoire à la boîte crânienne, vient exercer une pression locale sur le conduit auditif lors des cycles d'ouverture et de fermeture de la bouche. L'exploitation de cette source d'énergie a été explorée en 2012 par les chercheurs du laboratoire CRITIAS de l'École de Technologie  Supérieure de Montréal. Ces-derniers travaillent actuellement sur une meilleure caractérisation du comportement anatomique du conduit auditif. En parallèle, l'expertise dans le domaine de la récupération d'énergie au sein du laboratoire SYMME a motivé le projet de cette thèse qui s'est alors déroulé en étroite collaboration avec les chercheurs du laboratoire CRITIAS. Voici donc la structure des travaux menés afin de maximiser l'énergie exploitable depuis la déformation mécanique du conduit auditif : \\

Le premier chapitre présente les verrous technologiques des applications bio-intégrables et détaille les solutions existantes pour tenter de les résoudre. Il montre par ailleurs l'intérêt de la récupération d'énergie dans ce contexte et propose un aperçu des différentes sources d'énergie disponibles sur le corps humain, ainsi que les technologies de transduction existantes pour leur exploitation. Nous relevons notamment les défis associés, ainsi que les solutions qui existent à l'heure actuelle pour optimiser leur intégrabilité et leurs performances. Nous nous intéressons plus particulièrement à la récupération d'énergie dans le conduit auditif en considérant les stratégies existantes et en soulignant les aspects pouvant être améliorés dans les solutions que propose la littérature.
% LTeX: language=fr

Le second chapitre décrit la nouvelle architecture de récupérateur pour maximiser l'énergie récupérable depuis la déformation mécanique du conduit auditif. Celle-ci se compose essentiellement d'un bouchon d'oreille pressurisé pour capter l'énergie de déformation mécanique et la transmettre sous forme hydraulique en dehors du conduit auditif, d'un oscillateur bistable intégrant une structure de céramiques piézoélectriques pour augmenter la fréquence de l'énergie source et maximiser le rendement de conversion, ainsi que d'une solution innovante de valves de redirection hydraulique basée sur le flambement de tubes flexibles. Nous proposons une modélisation multiphysique du système et obtenons une première estimation de ses performances au travers d'un dimensionnement préliminaire.

Le troisième chapitre se concentre sur le dimensionnement, la conception, la fabrication et la caractérisation de l'oscillateur bistable. Une corrélation modèle-essais est menée pour valider le comportement théorique du convertisseur électromécanique.

Le quatrième chapitre présente une approche expérimentale pour le dimensionnement des valves hydrauliques à base de tubes flexibles. La viabilité de l'implémentation de telles valves dans le système est discutée au regard de leur cahier des charges établi au préalable.

Le cinquième chapitre propose une approche théorique pour le dimensionnement des valves hydrauliques en se basant sur des approximations géométriques et un modèle statique en éléments finis. Le modèle est corrélé avec les données expérimentales du chapitre précédent et les résultats sont discutés.

Enfin, le dernier chapitre est consacré au recalage du modèle système suite à l'intégration des données expérimentales issus des chapitres précédents. Le nouveau modèle démontre alors la faisabilité du récupérateur proposé, malgré un rendement de conversion expérimental plus faible que les estimations préliminaires. Les processus dissipatifs sont identifiés au sein de la chaîne de conversion et des solutions sont proposées pour les diminuer. De plus, l'influence des paramètres multiphysiques sur les performances du système est établie. Enfin, la fiabilité et la pertinence de la solution sont discutées et des pistes d'améliorations sont avancées pour le modèle, ainsi que pour une deuxième génération de prototype expérimental.


















