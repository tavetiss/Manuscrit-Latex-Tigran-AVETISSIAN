% LTeX: language=fr
%%%%%%%%%%%%%%%%%%%%%%%%%%%%%%%%%%%%%%%%%%%%%%%%%%%%%%%%%%%%%%%%%%%%%%%%%%
%%%%%                        Remerciements                          %%%%%%
%%%%%%%%%%%%%%%%%%%%%%%%%%%%%%%%%%%%%%%%%%%%%%%%%%%%%%%%%%%%%%%%%%%%%%%%%%

\phantomsection 
\addcontentsline{toc}{section}{Remerciements}
\addtocontents{toc}{\protect\addvspace{5pt}}

\vspace*{-1.6cm}
\begin{flushright}
\section*{\fontsize{20pt}{20pt}\lettrine[lines=1]{Remerciements}{}}
\end{flushright}
\vspace{0.5cm}


\lhead[\fancyplain{}{Remerciements}]
      {\fancyplain{}{}}
\chead[\fancyplain{}{}]
      {\fancyplain{}{}}
\rhead[\fancyplain{}{}]
      {\fancyplain{}{Remerciements}}
\lfoot[\fancyplain{}{}]
      {\fancyplain{}{}}
\cfoot[\fancyplain{}{\thepage}]
      {\fancyplain{}{\thepage}}
\rfoot[\fancyplain{}{}]%
     {\fancyplain{}{\scriptsize}}

%%%%%%%%%%%%%%%%%%%%%%%%%%%%%%%%%%%%%%%%%%%%%%%%%%%%%%%%%%%%%%%%%%%%%%%%%%
%%%%%                      Start part here                          %%%%%%
%%%%%%%%%%%%%%%%%%%%%%%%%%%%%%%%%%%%%%%%%%%%%%%%%%%%%%%%%%%%%%%%%%%%%%%%%%
% LTeX: language=fr

\emph{À travers ce premier chapitre de ma thèse, je voudrais adresser ma gratitude envers celles et ceux qui ont contribué, durant ces trois ans et demi, à un enrichissement remarquable de ma personne sur un plan de développement professionnel comme personnel.}\\

Avant toutes choses, je tiens à remercier le laboratoire SYMME, dirigé présentement par Ronan Le Dantec et précédemment par Georges Habchi, pour m'avoir accueilli et m'avoir mis à dispositions les moyens humains et matérie²²ls qui m'ont permis de réaliser les travaux de thèse présentés dans ce manuscrit. \\

Ce projet ambitieux et exploratoire s'est déroulé en collaboration avec le laboratoire CRITIAS, dirigé par le Pr. Jérémie voix. Je leur suis reconnaissant de m'avoir fourni les données biomécaniques sur lesquels se basent mes travaux, mais aussi pour m'avoir assisté lors de nombreuses réunions à distance et quelques rencontres fortement appréciées en présentiel. \\

Le bon déroulement de ces travaux a été possible avant tout grâce à mes deux mentors qui sont : mon Directeur de thèse Fabien Formosa et mon co-Directeur de thèse Adrien Badel. Bénéficier de vos compétences et de votre large expérience d'enseignants chercheurs a été un réel privilège pour moi. Je me suis toujours senti guidé lorsque j'étais dans l'impasse. Vous avez toujours su trouver la perspective pour me faire prendre le recul nécessaire sur mes travaux et m'aider à les valoriser. Vos regards complémentaires sur mon travail et vos visions de la recherche et de l'enseignement m'ont donné goût à ce que je fais et ont alimenté mes batteries mentales pour donner le meilleur de moi-même dans chacune de mes réalisations.

Vous côtoyer personnellement a aussi été un immense plaisir. J'ai beaucoup apprécié nos échanges enrichissants et la gaîté que vous y ameniez. Je me suis notamment senti réellement supporté par votre bienveillance et votre empathie dans les moments difficiles. Je conçois que l'encadrement de thèse ne soit pas une tâche évidente et je m'estime chanceux d'être tombé dans votre équipe !\\

Je tiens aussi à adresser mes remerciements aux membres du jury qui ont accepté d'évaluer mes travaux. Vos retours écrits et oraux ont été fortement appréciés et vos questions et remarques ont alimenté des discussions constructives très enrichissantes. \\

Ensuite, je voudrais dire un grand merci à tout le personnel administratif et technique du laboratoire SYMME. Merci à Nassika pour sa disponibilité et sa compréhension hors pair. Merci à Pascal et Jean-Christophe pour leur réactivité et leur le soutien matériel au quotidien. Merci à Françoise et Elza que j'ai pu côtoyer au secrétariat, vous avez été formidables dans votre accompagnement et toujours efficaces et arrangeantes. Enfin, un très grand merci à Blaise Girard sans qui mes bancs expérimentaux n'auraient jamais pu voir le jour !