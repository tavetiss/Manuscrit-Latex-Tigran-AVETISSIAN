%%%%%%%%%%%%%%%%%%%%%%%%%%%%%%%%%%%%%%%%%%%%%%%%%%%%%%%%%%%%%%%%%%%%%%%%%%
%%%%%                         CHAPITRE 1                            %%%%%%
%%%%%%%%%%%%%%%%%%%%%%%%%%%%%%%%%%%%%%%%%%%%%%%%%%%%%%%%%%%%%%%%%%%%%%%%%%

\lhead[\fancyplain{}{\leftmark}]%Pour les pages paires \bfseries
      {\fancyplain{}{}} %Pour les pages impaires
\chead[\fancyplain{}{}]%
      {\fancyplain{}{}}
\rhead[\fancyplain{}{}]%Pour les pages paires 
      {\fancyplain{}{\rightmark}}%Pour les pages impaires \bfseries
\lfoot[\fancyplain{}{}]%
      {\fancyplain{}{}}
\cfoot[\fancyplain{}{\thepage}]%\bfseries
      {\fancyplain{}{\thepage}} %\bfseries
\rfoot[\fancyplain{}{}]%
      {\fancyplain{}{\scriptsize}}
      

%%%%%%%%%%%%%%%%%%%%%%%%%%%%%%%%%%%%%%%%%%%%%%%%%%%%%%%%%%%%%%%%%%%%%%%%%%
%%%%%                      Start part here                          %%%%%%
\chapter{Introduction générale sur la récupération d’énergie}
\label{ch:1}
%%%%%%%%%%%%%%%%%%%%%%%%%%%%%%%%%%%%%%%%%%%%%%%%%%%%%%%%%%%%%%%%%%%%%%%%%%
\minitoc
\newpage
%/*\*/*\*/*\*/*\*/*\*/*\*/*\*/*\*/*\*/*\*/*\*/*\*/*\*/*\*/*\*/*\*/*\*/*\
\section{Introduction}
\label{sec:1.intro}
%/*\*/*\*/*\*/*\*/*\*/*\*/*\*/*\*/*\*/*\*/*\*/*\*/*\*/*\*/*\*/*\*/*\*/*\
\lettrine[lines=1]{I}{ }ntroduction générale \cite{Bob2000} \cite{Agashe2008}

%/*\*/*\*/*\*/*\*/*\*/*\*/*\*/*\*/*\*/*\*/*\*/*\*/*\*/*\*/*\*/*\*/*\*/*\
\section{Pourquoi faire de la récupération d’énergie}
\label{sec:1.1}
%/*\*/*\*/*\*/*\*/*\*/*\*/*\*/*\*/*\*/*\*/*\*/*\*/*\*/*\*/*\*/*\*/*\*/*\

%++++++++++++++++++++++++++++++++
 \subsection{Appareils électroniques nomades}
 \label{subsec:1.1.1}
%++++++++++++++++++++++++++++++++

%++++++++++++++++++++++++++++++++
 \subsection{Longévité des batteries}
\label{subsec:1.1.2}
%++++++++++++++++++++++++++++++++

%/*\*/*\*/*\*/*\*/*\*/*\*/*\*/*\*/*\*/*\*/*\*/*\*/*\*/*\*/*\*/*\*/*\*/*\
\section{Les types de gisements énergétiques existants}
\label{sec:1.2}
%/*\*/*\*/*\*/*\*/*\*/*\*/*\*/*\*/*\*/*\*/*\*/*\*/*\*/*\*/*\*/*\*/*\*/*\

%++++++++++++++++++++++++++++++++
\section{Verrous technologiques pour l’exploitation des sources d’énergies}
\label{sec:1.3}
%++++++++++++++++++++++++++++++++

%++++++++++++++++++++++++++++++++
\section{Les grandes familles de récupérateurs d’énergie}
\label{sec:1.4}
%++++++++++++++++++++++++++++++++



%
%
%\subsection{Figures}
%
%\begin{figure}
%	\centering
%	\def\svgwidth{0.5\columnwidth}
%	\fontsize{10pt}{10pt}\selectfont\input{../Chap1/Figure/grande_simple.pdf_tex}
%	\caption{Figure simple}\label{fig:figure_simple}
%\end{figure}
%
%\begin{figure}
%	\centering
%	\begin{subfigure}[b]{0.45\textwidth}
%		\centering
%		\def\svgwidth{\columnwidth}
%		\fontsize{10pt}{10pt}\selectfont\input{../Chap1/Figure/double.pdf_tex}
%		\caption{double (a)}\label{fig:double_a}
%	\end{subfigure}
%	\qquad
%	\begin{subfigure}[b]{0.45\textwidth}
%		\centering
%		\def\svgwidth{\columnwidth}
%		\fontsize{10pt}{10pt}\selectfont\input{../Chap1/Figure/double.pdf_tex}
%		\caption{double (b)}\label{fig:double_b}
%	\end{subfigure}
%	\caption{Figure double}\label{fig:double_figure}
%\end{figure}
%
%
%\begin{figure}
%	\centering
%	\begin{subfigure}[b]{0.329\textwidth}
%		\centering
%		\def\svgwidth{\columnwidth}
%		\fontsize{10pt}{10pt}\selectfont\input{../Chap1/Figure/triple.pdf_tex}
%		\caption{triple (a)}\label{fig:triple_a}
%	\end{subfigure}
%	\begin{subfigure}[b]{0.329\textwidth}
%		\centering
%		\def\svgwidth{\columnwidth}
%		\fontsize{10pt}{10pt}\selectfont\input{../Chap1/Figure/triple.pdf_tex}
%		\caption{triple (b)}\label{fig:triple_b}
%	\end{subfigure}
%	\begin{subfigure}[b]{0.329\textwidth}
%		\centering
%		\def\svgwidth{\columnwidth}
%		\fontsize{10pt}{10pt}\selectfont\input{../Chap1/Figure/triple.pdf_tex}
%		\caption{triple (c)}\label{fig:triple_c}
%	\end{subfigure}
%	\caption{Figure triple}\label{fig:triple_figure}
%	
%\end{figure}
%
%\newpage 
%
%\subsection{Tableaux}
%\begin{table}
%\centering
%\rowcolors[]{2}{black!8}{}{
%\begin{tabular}{l||cccc}
%\rowcolor{blue!10}
%N\textsuperscript{o} & A & B & C & D\\
%\cmidrule[1pt]{1-5}
%1 & x & x & 0 & 0\\
%2 & 0 & x & 0 & 0\\
%3 & x & 0 & x & 0\\
%4 & x & x & 0 & x\\
%\cmidrule[1pt]{1-5}
%\end{tabular}}
%\caption{Tableau}\label{tbl:Tableau}
%\end{table} 
