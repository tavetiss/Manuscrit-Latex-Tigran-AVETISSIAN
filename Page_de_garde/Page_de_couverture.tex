%Ltex:language=fr
\Sethpageshiftt{11mm}   
\Setvpageshiftt{13mm}   
\vspace{0mm}
% \Auteur{Tigran AVETISSIAN}
\Resume{
Ces travaux visent à récupérer l’énergie ambiante pour améliorer l'autonomie des batteries rechargeables des appareils d'aide à l'audition. La déformation mécanique du conduit auditif suite aux mouvements de la mâchoire représente ici une source d’énergie mécanique potentiellement exploitable.

On propose une nouvelle architecture de récupérateur d’énergie pour répondre au besoin. Le système se compose d’un bouchon d’oreille rempli d’un fluide incompressible, agissant ainsi comme une mini-pompe sous la déformation imposée par le conduit auditif. Cette énergie hydraulique est transmise à un oscillateur mécanique bistable au travers de deux circuits hydrauliques intégrant chacune un piston et une valve hydraulique s’actionnant alternativement à chaque fermeture de mâchoire. Les valves, exploitant le flambement de tubes en flexion, sont commutées par le mouvement de l’oscillateur bistable, rendant ainsi son mouvement cyclique automatique. Un transducteur piézoélectrique est alors utilisé pour convertir l’énergie transmise depuis le bouchon d’oreille à l'oscillateur.

Une modélisation du comportement multiphysique du système est établie et le rendement théorique est de 67\%. Un modèle théorique est aussi proposé pour le dimensionnement des valves. Trois bancs d’essais sont mis en \oe{}uvre pour caractériser expérimentalement le convertisseur électromécanique et le comportement statique et hydraulique des valves. Les corrélations modèle-essais valident les résultats du modèle théorique du convertisseur et la réaction statique des valves. Des solutions sont alors envisagées pour améliorer le rendement expérimental du convertisseur et les aspects prédictifs du modèle théorique des valves.

Le comportement expérimental des différents éléments est intégré dans le modèle théorique global du système et les résultats sont discutés, en dégageant notamment l'influence des différents paramètres sur les performances du système. Des pistes d’améliorations sont finalement proposées pour réduire les pertes énergétiques dans la chaîne de conversion et faciliter l’intégration dans le contexte applicatif.
}

\Motscles{Récupération d'énergie, intra-auriculaire, frequency-up, système multiphysique, valves hydrauliques}

%Ltex:language=en
\Abstr{This work aims at harvesting ambient energy to improve the autonomy of rechargeable batteries of hearing aids. The mechanical deformation of the ear canal induced by the jaw movements represents here a potentially exploitable source of mechanical energy.

A new architecture of energy harvester is then proposed. The system is composed of an earplug filled with an incompressible fluid, acting as a mini-pump under the deformation imposed by the ear canal. This hydraulic energy is transmitted to a bistable mechanical oscillator through two hydraulic circuits, each integrating a piston and a hydraulic valve that are alternately activated at each jaw closure. The valves, exploiting the buckling of bent tubes, are switched by the bistable oscillator movement, making its cyclic motion automatic. A piezoelectric transducer is then used to convert the energy transmitted from the earplug to the oscillator.

A model of the multiphysic behavior of the system is established, and the theoretical efficiency is 67\%. A theoretical model is also proposed for the valves design. Three test benches are developed to experimentally characterize the electromechanical converter and the valves static and hydraulic behavior. The model-test correlations support the converter theoretical model results and the valves theoretical static response. Solutions are then proposed to improve the fabricated converter performances and the theoretical predictions of the valves hydraulic behavior.

The experimental behavior of the different elements is integrated in the global theoretical model of the system and the results are discussed by highlighting the influence of the different parameters on the system performance. Possible improvements are finally proposed to reduce the energy losses in the conversion chain and to facilitate the integration in the application environment.
}   

\Keywords{Energy harvesting, in-ear, frequency-up, multiphysic system, hydraulic valves}

\MakeUGthesePDC    

    
